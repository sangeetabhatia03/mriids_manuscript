%\documentclass[bgd, online, hvmath]{copernicus_discussions}
\documentclass[bgd, online, hvmath]{style/copernicus}
%\documentclass[hess, online, hvmath]{copernicus}

\usepackage{booktabs}
\usepackage{longtable}
%\usepackage{lmodern}
%\usepackage{amssymb,amsmath}
\usepackage{fixltx2e} % provides \textsubscript


\usepackage{natbib}
\bibliographystyle{apalike}

\usepackage{color}
%\usepackage[pdftex]{hyperref}
\usepackage[colorlinks=true, linkcolor=black, citecolor=blue, urlcolor=black]{hyperref}
\usepackage{pifont}
\usepackage{verbatim}
\usepackage{textcomp}
\usepackage{amsmath}
\usepackage{amssymb}
\usepackage[singlelinecheck=false]{caption}
\usepackage[figuresright]{rotating}
\usepackage{float}
\graphicspath{{pic/}}

\begin{document}\sloppy

\title{Big Brother Is Watching : Using Digital Surveillance Tools for Near
Real-Time Mapping of the Risk of International Infectious Disease Spread}

\author{Sangeeta Bhatia, Anne Cori and Pierre Nouvellet}


%\author[]{NAME}
%\author[]{NAME}
%\author[]{NAME}

%\affil[]{ADDRESS}
%\affil[]{ADDRESS}

%\affil[1]{Department of Micrometeorology, University of Bayreuth, Germany}
%\affil[2]{Member of Bayreuth Center of Ecology and Environmental Research
%(BayCEER), University of Bayreuth, Germany}
%\correspondence{P.~Zhao~(peng.zhao@uni-bayreuth.de), J.~L\"{u}ers~(johannes.lueers@uni-bayreuth.de)}
%\runningtitle{short title}
%\runningauthor{authors}
%\received{23 January 2012}
%\accepted{28 February 2012}
%\published{}


\firstpage{1}
\maketitle


In our increasingly interconnected world, it is crucial to understand
the risk of an outbreak originating in one country/region and spreading
to the rest of the world. Rapid recognition and response to potential
pandemics and emerging diseases have become essential global health
priorities. Digital disease surveillance tools such as ProMed and
HealthMap have the potential to serve as important early warning systems
as well as complement the field surveillance data during an ongoing
outbreak. While there are a number of systems that carry out digital
disease surveillance, there is as yet a lack of tools that can compile
and analyse the generated data to produce easily understood actionable
reports. We present a flexible statistical model that uses different
streams of data (such as disease surveillance data, mobility data etc.)
for short-term incidence trend forecasting.\\
In validating the model using data collected by ProMED and HealthMap
during the 2014-2016 West African Ebola outbreak, we provide a realistic
appraisal of the strengths and limitations of such data in incidence
forecasting. We infer incidence trends at finer spatial scales from
aggregated data. Our work shows how the data from event based
surveillance systems (EBS) can complement the data collected from
traditional public health infrastructure. During an ongoing crisis,
combining data from different sources gives stakeholders a more complete
picture.

\section{Introduction}\label{introduction}

other tools that do similar stuff - EpiDMS \citep{liu2016epidms}
slightly old paper - authors curated news themselves!
\citep{chowell2016elucidating}


\bibliography{bib/mriids.bib}

\end{document}
