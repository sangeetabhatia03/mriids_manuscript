\documentclass[9pt,twoside,lineno]{pnas-new}
\templatetype{pnassupportinginfo}
\title{Using Digital Surveillance Tools for Near Real-Time Mapping of
  the Risk of International Infectious Disease Spread: Ebola as a Case
  Study}

\author{Sangeeta Bhatia, Britta Lassmann, Emily Cohn, Malwina Carrion,
Moritz Kraemer, Mark Herringer, John Brownstein, Larry Madoff, Anne
Cori, Pierre Nouvellet}
\correspondingauthor{Sangeeta Bhatia.\\E-mail: s.bhatia@imperial.ac.uk}

% \usepackage[]{mathpazo}
\usepackage{amssymb,amsmath, booktabs}
% \usepackage[T1]{fontenc}
% \usepackage[utf8]{inputenc}
 \usepackage{graphicx,grffile}

\usepackage{caption}
\usepackage{subcaption}
% \graphicspath{{/Volumes/Sangeeta EHD/mriids_manuscript/ms-figures/si-figures}}
\renewcommand{\figurename}{Fig SI}
%\usepackage{hyperref}
\begin{document}
\maketitle

%% Adds the main heading for the SI text. Comment out this line if you do not have any supporting information text.
\SItext

 
\tableofcontents


\section*{Overview}\label{overview}

In this supplement, we present the details of the pre-processing of
ProMED and HealthMap feeds \ref{sec:data-cleaning}, 
the model results using ProMED,
HealthMap and WHO data and the impact of the datasources and model
parameters on the performance of the model. We varied the length of the
time window used for model fitting (see Methods for details). SI Sections
2, SI Section 3 and SI Section 4 present the forecasts using ProMED,
HealthMap and WHO data respectively using different calibration windows (2, 4
and 6 weeks) and forecast horizons (4, 6 and 8 weeks). The calibration window
did not have a strong influence on the model performance (SI Figure 28).
Similarly, the data used also did not influence the model performance
(SI Figure 29). We explored the impact of alternate priors for the
gravity model parameter on the results from the model. The results of
this sensitivity analysis are presented in SI Section 7. We present
additional analysis on the predicting the spatial spread of the
epidemic in Section~\ref{sec:spatial-spread}.


\section{Data cleaning}\label{sec:data-cleaning}

The data from ProMED and HealthMap was pre-processed to remove
inconsistencies and address missing data (see Methods). The data on
cumulative number of confirmed and suspected Ebola cases in Guinea,
Liberia, Sierra Leone, Senegal, Mali, Nigeria and Ghana were extracted
from ProMED and HealthMap feeds. The pre-processing workflow is 
illustrated using data for Sierra Leone from ProMED.

\begin{figure}
  \centering
  \includegraphics[width = \textwidth]{/Volumes/Sangeeta EHD/mriids_manuscript/ms-figures/si-figures/ProMED/27062019-promed-loglinear-SierraLeone.pdf}
  \caption{Illustration of workflow for Processing ProMED feed.
    Raw ProMED feed consisted of suspected and confirmed
    cases and suspected and confirmed
    deaths. The top left figure (a) shows the suspected (blue) and
    confirmed (red) cases in Sierra Leone in the raw feed.
    We used the suspected and confirmed cases to derive cumulative
    incidence data (b). Duplicate alerts on a day were then removed
    by assigning the median of the cases reported to this day.
    (c).
    If there were outliers in the data, we removed them in the next step (d).
    We then made the cumulative case count monotonically increasing (e) by
    removing inconsistent records. Finally, missing data was imputed using
    log-linear interpolation (f). Imputed points are show in light blue.
    See Methods for details.}
\label{fig:dataclean}
\end{figure}\FloatBarrier

\subsection{Weekly incidence from ProMED, HealthMap and WHO data}

We processed the data from ProMED and HealthMap and derived daily and
weekly incidence. The weekly incidence series derived from the three
data sources were highly correlated.

\begin{figure}
  \centering
  \includegraphics[width = \textwidth]{/Volumes/Sangeeta EHD/mriids_manuscript/ms-figures/si-figures/other/2019-10-29_weekly_incid_14_28.pdf}
  \caption{Comparison of national weekly incidene trends from ProMED (blue),
    HealthMap (green) and WHO (yellow) data for Guinea, Liberia and
    Sierra Leone. Weeks for which all daily data points were imputed
    are shown in lighter shades of blue and green respectively. The
y-axis differs for each plot.}
\label{fig:weekly}
\end{figure}\FloatBarrier


\subsection{Correlation between R estimates}\label{correlation-between-r-estimates}

The correlation between estimates of time-varying reproduction number 
estimated using ProMED, HealthMap and WHO data depended on the
time window used for estimation and the country.

\begin{figure}
  \centering
  \includegraphics[width=\textwidth]{/Volumes/Sangeeta EHD/mriids_manuscript/ms-figures/si-figures/other/2019-10-25_median_R_corr.pdf}
  \caption{Correlation between time-varying reproduction number estimated from
    ProMED, HealthMap and WHO data. The reporduction numbers were
    estimated using R package EpiEstim over a 2, 4 or 6 week sliding
    window. Median estimates from WHO data are on the x-axis and the
    median estimates using ProMED (blue) and HealthMap (green) data
    are on the y-axis. All correlation coefficients were statistically
  significant.}
  \label{fig:rcorr}
\end{figure}\FloatBarrier


\begin{figure}
  \centering
  \includegraphics[width=\textwidth]{/Volumes/Sangeeta EHD/mriids_manuscript/ms-figures/si-figures/other/2019-10-29_r_real_retro_correlation.pdf}
  \caption{Correlation between time-varying reproduction number estimated from
    ProMED, HealthMap and WHO data. The reporduction numbers were
    estimated using R package EpiEstim over a 2, 4 or 6 week sliding
    window. Median estimates from WHO data are on the x-axis and the
    median estimates using ProMED (blue) and HealthMap (green) data
    are on the y-axis. All correlation coefficients were statistically
  significant.}
  \label{fig:rcorrrealretro}
\end{figure}\FloatBarrier




\section{Forecasts using ProMED data}\label{sec:pm} 

This section presents the forecast produced using ProMED data with
calibration window of 2 (Section~\ref{sec:pm2}), 
4 (Section~\ref{sec:pm4}) and 6 
(Section~\ref{sec:pm6}) weeks over a 
forecast horizon of 4, 6 and 8 weeks. Results using calibration window
of 2 weeks and forecast horizon of 4 weeks are presented in the main
text.

\subsection{Calibration window of 2 weeks}\label{sec:pm2}
\subsubsection{Forecast horozon of 4 weeks}\label{sec:pm24}

\begin{table}
  \centering
  \caption{Percentage of weeks with observed incidence in 95\% forecast
    interval. In forecasting ahead, we assumed transmissibility to be
    constant over the forecast horizon. If the 97.5th percentile of the R
    estimate used for forecasting was less than 1, we defined the epidemic
    to be in the declining phase during this period. Similarly, if the 2.5th
    percentile of R was greater than 1, we defined the epidemic to be in a
    growing phase. The phase was set to stable where the 95\% Credible
    Interval of the R estimates contained 1.}
  \label{tab:propinci}
  \begin{tabular}{lllll}
    \toprule
    Country  & Growing & Declining & Stable & Overall \\
    \midrule
    Guinea  & 44.0\% & 42.2\% & 71.9\% & 54.4\% \\
    Liberia & 18.3\% & 31.5\% & 69.5\% & 49.3\% \\
    Sierra Leone & 25.9\% & 46.0\% & 55.7\% & 42.5\% \\
    All & 30.8\% & 40.2\% & 66.7\% & 48.7\% \\
    \bottomrule
  \end{tabular}
\end{table}

\subsubsection{Forecase horizon of 6 weeks}\label{sec:pm26} 

\begin{figure}\centering
\includegraphics[width = \textwidth]{{../../ms-figures/si-figures/ProMED/projections-viz-fixed-country-15-1.ProMED_14_42}.pdf}
\caption{Observed and predicted
  incidence, and reproduction number estimates from ProMED data. The top
  panel shows the weekly incidence derived from ProMED data and the 2
  weeks incidence forecast on log scale. The solid dots represent the
  observed weekly incidence where the light blue
  dots show weeks for which
  all data points were obtained using interpolation. The
  projections are made over 6 week windows, based on the reproduction
  number estimated in the previous 2 weeks. The middle figure in each
  panel shows the reproduction number used to make forecasts over each 6
  week forecast horizon. The bottom figure shows the effective
  reproduction number estimated retrospectively using the full dataset up
  to the length of one calibration window before the end.
  In each case, the solid black line is the median estimate and
  the shaded region represents the 95\% Credible Interval. The red
  horizontal dashed line indicates the $R_t = 1$ threshold. Results are
  shown for the three mainly affected countries although the analysis was
  done jointly using data for all countries in Africa.}
\label{fig:pm26}
\end{figure}\FloatBarrier

\subsubsection{Forecase horizon of 8 weeks}\label{sec:pm28} 

\begin{figure}\centering
\includegraphics[width = \textwidth]{{../../ms-figures/si-figures/ProMED/projections-viz-fixed-country-15-1.ProMED_14_56}.pdf}
\caption{Observed and predicted
  incidence, and reproduction number estimates from ProMED data. The top
  panel shows the weekly incidence derived from ProMED data and the 2
  weeks incidence forecast on log scale. The solid dots represent the
  observed weekly incidence where the light blue
  dots show weeks for which
  all data points were obtained using interpolation. The
  projections are made over 8 week windows, based on the reproduction
  number estimated in the previous 2 weeks. The middle figure in each
  panel shows the reproduction number used to make forecasts over each 8
  week forecast horizon. The bottom figure shows the effective
  reproduction number estimated retrospectively using the full dataset up
  to the length of one calibration window before the end.
  In each case, the solid black line is the median estimate and
  the shaded region represents the 95\% Credible Interval. The red
  horizontal dashed line indicates the $R_t = 1$ threshold. Results are
  shown for the three mainly affected countries although the analysis was
  done jointly using data for all countries in Africa.}
\label{fig:pm28}
\end{figure}\FloatBarrier


\subsection{Calibration window of 4 weeks}\label{sec:pm4}
\subsubsection{Forecase horizon of 4 weeks}\label{sec:pm44} 

\begin{figure}\centering
\includegraphics[width = \textwidth]{{../../ms-figures/si-figures/ProMED/projections-viz-fixed-country-15-1.ProMED_28_28}.pdf}
\caption{Observed and predicted
  incidence, and reproduction number estimates from ProMED data. The top
  panel shows the weekly incidence derived from ProMED data and the 4
  weeks incidence forecast on log scale. The solid dots represent the
  observed weekly incidence where the light blue
  dots show weeks for which
  all data points were obtained using interpolation. The
  projections are made over 4 week windows, based on the reproduction
  number estimated in the previous 4 weeks. The middle figure in each
  panel shows the reproduction number used to make forecasts over each 4
  week forecast horizon. The bottom figure shows the effective
  reproduction number estimated retrospectively using the full dataset up
  to the length of one calibration window before the end.
  In each case, the solid black line is the median estimate and
  the shaded region represents the 95\% Credible Interval. The red
  horizontal dashed line indicates the $R_t = 1$ threshold. Results are
  shown for the three mainly affected countries although the analysis was
  done jointly using data for all countries in Africa.}
\label{fig:pm44}
\end{figure}\FloatBarrier

\subsubsection{Forecase horizon of 6 weeks}\label{sec:pm46} 

\begin{figure}\centering
\includegraphics[width = \textwidth]{{../../ms-figures/si-figures/ProMED/projections-viz-fixed-country-15-1.ProMED_28_42}.pdf}
\caption{Observed and predicted
  incidence, and reproduction number estimates from ProMED data. The top
  panel shows the weekly incidence derived from ProMED data and the 4
  weeks incidence forecast on log scale. The solid dots represent the
  observed weekly incidence where the light blue
  dots show weeks for which
  all data points were obtained using interpolation. The
  projections are made over 6 week windows, based on the reproduction
  number estimated in the previous 4 weeks. The middle figure in each
  panel shows the reproduction number used to make forecasts over each 6
  week forecast horizon. The bottom figure shows the effective
  reproduction number estimated retrospectively using the full dataset up
  to the length of one calibration window before the end.
  In each case, the solid black line is the median estimate and
  the shaded region represents the 95\% Credible Interval. The red
  horizontal dashed line indicates the $R_t = 1$ threshold. Results are
  shown for the three mainly affected countries although the analysis was
  done jointly using data for all countries in Africa.}
\label{fig:pm46}
\end{figure}\FloatBarrier

\subsubsection{Forecase horizon of 8 weeks}\label{sec:pm48} 

\begin{figure}\centering
\includegraphics[width = \textwidth]{{../../ms-figures/si-figures/ProMED/projections-viz-fixed-country-15-1.ProMED_28_56}.pdf}
\caption{Observed and predicted
  incidence, and reproduction number estimates from ProMED data. The top
  panel shows the weekly incidence derived from ProMED data and the 4
  weeks incidence forecast on log scale. The solid dots represent the
  observed weekly incidence where the light blue
  dots show weeks for which
  all data points were obtained using interpolation. The
  projections are made over 8 week windows, based on the reproduction
  number estimated in the previous 4 weeks. The middle figure in each
  panel shows the reproduction number used to make forecasts over each 8
  week forecast horizon. The bottom figure shows the effective
  reproduction number estimated retrospectively using the full dataset up
  to the length of one calibration window before the end.
  In each case, the solid black line is the median estimate and
  the shaded region represents the 95\% Credible Interval. The red
  horizontal dashed line indicates the $R_t = 1$ threshold. Results are
  shown for the three mainly affected countries although the analysis was
  done jointly using data for all countries in Africa.}
\label{fig:pm48}
\end{figure}\FloatBarrier


\subsection{Calibration window of 6 weeks}\label{sec:pm6}
\subsubsection{Forecase horizon of 4 weeks}\label{sec:pm64} 

\begin{figure}\centering
\includegraphics[width = \textwidth]{{../../ms-figures/si-figures/ProMED/projections-viz-fixed-country-15-1.ProMED_42_28}.pdf}
\caption{Observed and predicted
  incidence, and reproduction number estimates from ProMED data. The top
  panel shows the weekly incidence derived from ProMED data and the 6
  weeks incidence forecast on log scale. The solid dots represent the
  observed weekly incidence where the light blue
  dots show weeks for which
  all data points were obtained using interpolation. The
  projections are made over 4 week windows, based on the reproduction
  number estimated in the previous 6 weeks. The middle figure in each
  panel shows the reproduction number used to make forecasts over each 4
  week forecast horizon. The bottom figure shows the effective
  reproduction number estimated retrospectively using the full dataset up
  to the length of one calibration window before the end.
  In each case, the solid black line is the median estimate and
  the shaded region represents the 95\% Credible Interval. The red
  horizontal dashed line indicates the $R_t = 1$ threshold. Results are
  shown for the three mainly affected countries although the analysis was
  done jointly using data for all countries in Africa.}
\label{fig:pm64}
\end{figure}\FloatBarrier

\subsubsection{Forecase horizon of 6 weeks}\label{sec:pm66} 

\begin{figure}\centering
\includegraphics[width = \textwidth]{{../../ms-figures/si-figures/ProMED/projections-viz-fixed-country-15-1.ProMED_42_42}.pdf}
\caption{Observed and predicted
  incidence, and reproduction number estimates from ProMED data. The top
  panel shows the weekly incidence derived from ProMED data and the 6
  weeks incidence forecast on log scale. The solid dots represent the
  observed weekly incidence where the light blue
  dots show weeks for which
  all data points were obtained using interpolation. The
  projections are made over 6 week windows, based on the reproduction
  number estimated in the previous 6 weeks. The middle figure in each
  panel shows the reproduction number used to make forecasts over each 6
  week forecast horizon. The bottom figure shows the effective
  reproduction number estimated retrospectively using the full dataset up
  to the length of one calibration window before the end.
  In each case, the solid black line is the median estimate and
  the shaded region represents the 95\% Credible Interval. The red
  horizontal dashed line indicates the $R_t = 1$ threshold. Results are
  shown for the three mainly affected countries although the analysis was
  done jointly using data for all countries in Africa.}
\label{fig:pm66}
\end{figure}\FloatBarrier

\subsubsection{Forecase horizon of 8 weeks}\label{sec:pm68} 

\begin{figure}\centering
\includegraphics[width = \textwidth]{{../../ms-figures/si-figures/ProMED/projections-viz-fixed-country-15-1.ProMED_42_56}.pdf}
\caption{Observed and predicted
  incidence, and reproduction number estimates from ProMED data. The top
  panel shows the weekly incidence derived from ProMED data and the 6
  weeks incidence forecast on log scale. The solid dots represent the
  observed weekly incidence where the light blue
  dots show weeks for which
  all data points were obtained using interpolation. The
  projections are made over 8 week windows, based on the reproduction
  number estimated in the previous 6 weeks. The middle figure in each
  panel shows the reproduction number used to make forecasts over each 8
  week forecast horizon. The bottom figure shows the effective
  reproduction number estimated retrospectively using the full dataset up
  to the length of one calibration window before the end.
  In each case, the solid black line is the median estimate and
  the shaded region represents the 95\% Credible Interval. The red
  horizontal dashed line indicates the $R_t = 1$ threshold. Results are
  shown for the three mainly affected countries although the analysis was
  done jointly using data for all countries in Africa.}
\label{fig:pm68}
\end{figure}\FloatBarrier

\section{Forecasts using HealthMap data}\label{sec:hm}

This section presents the forecasts over 4, 6 and 8 weeks produced
using HealthMap data and calibration window of 2, 4 and 6 weeks.

\subsection{Calibration window of 2 weeks}\label{sec:hm2}
\subsubsection{Forecase horizon of 4 weeks}\label{sec:hm24} 

\begin{figure}\centering
\includegraphics[width = \textwidth]{{../../ms-figures/si-figures/HealthMap/projections-viz-fixed-country-15-1.HealthMap_14_28}.pdf}
\caption{Observed and predicted
  incidence, and reproduction number estimates from HealthMap data. The top
  panel shows the weekly incidence derived from HealthMap data and the 2
  weeks incidence forecast on log scale. The solid dots represent the
  observed weekly incidence where the light green
  dots show weeks for which
  all data points were obtained using interpolation. The
  projections are made over 4 week windows, based on the reproduction
  number estimated in the previous 2 weeks. The middle figure in each
  panel shows the reproduction number used to make forecasts over each 4
  week forecast horizon. The bottom figure shows the effective
  reproduction number estimated retrospectively using the full dataset up
  to the length of one calibration window before the end.
  In each case, the solid black line is the median estimate and
  the shaded region represents the 95\% Credible Interval. The red
  horizontal dashed line indicates the $R_t = 1$ threshold. Results are
  shown for the three mainly affected countries although the analysis was
  done jointly using data for all countries in Africa.}
\label{fig:hm24}
\end{figure}\FloatBarrier

\subsubsection{Forecase horizon of 6 weeks}\label{sec:hm26} 

\begin{figure}\centering
\includegraphics[width = \textwidth]{{../../ms-figures/si-figures/HealthMap/projections-viz-fixed-country-15-1.HealthMap_14_42}.pdf}
\caption{Observed and predicted
  incidence, and reproduction number estimates from HealthMap data. The top
  panel shows the weekly incidence derived from HealthMap data and the 2
  weeks incidence forecast on log scale. The solid dots represent the
  observed weekly incidence where the light green
  dots show weeks for which
  all data points were obtained using interpolation. The
  projections are made over 6 week windows, based on the reproduction
  number estimated in the previous 2 weeks. The middle figure in each
  panel shows the reproduction number used to make forecasts over each 6
  week forecast horizon. The bottom figure shows the effective
  reproduction number estimated retrospectively using the full dataset up
  to the length of one calibration window before the end.
  In each case, the solid black line is the median estimate and
  the shaded region represents the 95\% Credible Interval. The red
  horizontal dashed line indicates the $R_t = 1$ threshold. Results are
  shown for the three mainly affected countries although the analysis was
  done jointly using data for all countries in Africa.}
\label{fig:hm26}
\end{figure}\FloatBarrier

\subsubsection{Forecase horizon of 8 weeks}\label{sec:hm28} 

\begin{figure}\centering
\includegraphics[width = \textwidth]{{../../ms-figures/si-figures/HealthMap/projections-viz-fixed-country-15-1.HealthMap_14_56}.pdf}
\caption{Observed and predicted
  incidence, and reproduction number estimates from HealthMap data. The top
  panel shows the weekly incidence derived from HealthMap data and the 2
  weeks incidence forecast on log scale. The solid dots represent the
  observed weekly incidence where the light green
  dots show weeks for which
  all data points were obtained using interpolation. The
  projections are made over 8 week windows, based on the reproduction
  number estimated in the previous 2 weeks. The middle figure in each
  panel shows the reproduction number used to make forecasts over each 8
  week forecast horizon. The bottom figure shows the effective
  reproduction number estimated retrospectively using the full dataset up
  to the length of one calibration window before the end.
  In each case, the solid black line is the median estimate and
  the shaded region represents the 95\% Credible Interval. The red
  horizontal dashed line indicates the $R_t = 1$ threshold. Results are
  shown for the three mainly affected countries although the analysis was
  done jointly using data for all countries in Africa.}
\label{fig:hm28}
\end{figure}\FloatBarrier


\subsection{Calibration window of 4 weeks}\label{sec:hm4}
\subsubsection{Forecase horizon of 4 weeks}\label{sec:hm44} 

\begin{figure}\centering
\includegraphics[width = \textwidth]{{../../ms-figures/si-figures/HealthMap/projections-viz-fixed-country-15-1.HealthMap_28_28}.pdf}
\caption{Observed and predicted
  incidence, and reproduction number estimates from HealthMap data. The top
  panel shows the weekly incidence derived from HealthMap data and the 4
  weeks incidence forecast on log scale. The solid dots represent the
  observed weekly incidence where the light green
  dots show weeks for which
  all data points were obtained using interpolation. The
  projections are made over 4 week windows, based on the reproduction
  number estimated in the previous 4 weeks. The middle figure in each
  panel shows the reproduction number used to make forecasts over each 4
  week forecast horizon. The bottom figure shows the effective
  reproduction number estimated retrospectively using the full dataset up
  to the length of one calibration window before the end.
  In each case, the solid black line is the median estimate and
  the shaded region represents the 95\% Credible Interval. The red
  horizontal dashed line indicates the $R_t = 1$ threshold. Results are
  shown for the three mainly affected countries although the analysis was
  done jointly using data for all countries in Africa.}
\label{fig:hm44}
\end{figure}\FloatBarrier

\subsubsection{Forecase horizon of 6 weeks}\label{sec:hm46} 

\begin{figure}\centering
\includegraphics[width = \textwidth]{{../../ms-figures/si-figures/HealthMap/projections-viz-fixed-country-15-1.HealthMap_28_42}.pdf}
\caption{Observed and predicted
  incidence, and reproduction number estimates from HealthMap data. The top
  panel shows the weekly incidence derived from HealthMap data and the 4
  weeks incidence forecast on log scale. The solid dots represent the
  observed weekly incidence where the light green
  dots show weeks for which
  all data points were obtained using interpolation. The
  projections are made over 6 week windows, based on the reproduction
  number estimated in the previous 4 weeks. The middle figure in each
  panel shows the reproduction number used to make forecasts over each 6
  week forecast horizon. The bottom figure shows the effective
  reproduction number estimated retrospectively using the full dataset up
  to the length of one calibration window before the end.
  In each case, the solid black line is the median estimate and
  the shaded region represents the 95\% Credible Interval. The red
  horizontal dashed line indicates the $R_t = 1$ threshold. Results are
  shown for the three mainly affected countries although the analysis was
  done jointly using data for all countries in Africa.}
\label{fig:hm46}
\end{figure}\FloatBarrier

\subsubsection{Forecase horizon of 8 weeks}\label{sec:hm48} 

\begin{figure}\centering
\includegraphics[width = \textwidth]{{../../ms-figures/si-figures/HealthMap/projections-viz-fixed-country-15-1.HealthMap_28_56}.pdf}
\caption{Observed and predicted
  incidence, and reproduction number estimates from HealthMap data. The top
  panel shows the weekly incidence derived from HealthMap data and the 4
  weeks incidence forecast on log scale. The solid dots represent the
  observed weekly incidence where the light green
  dots show weeks for which
  all data points were obtained using interpolation. The
  projections are made over 8 week windows, based on the reproduction
  number estimated in the previous 4 weeks. The middle figure in each
  panel shows the reproduction number used to make forecasts over each 8
  week forecast horizon. The bottom figure shows the effective
  reproduction number estimated retrospectively using the full dataset up
  to the length of one calibration window before the end.
  In each case, the solid black line is the median estimate and
  the shaded region represents the 95\% Credible Interval. The red
  horizontal dashed line indicates the $R_t = 1$ threshold. Results are
  shown for the three mainly affected countries although the analysis was
  done jointly using data for all countries in Africa.}
\label{fig:hm48}
\end{figure}\FloatBarrier


\subsection{Calibration window of 6 weeks}\label{sec:hm6}
\subsubsection{Forecase horizon of 4 weeks}\label{sec:hm64} 

\begin{figure}\centering
\includegraphics[width = \textwidth]{{../../ms-figures/si-figures/HealthMap/projections-viz-fixed-country-15-1.HealthMap_42_28}.pdf}
\caption{Observed and predicted
  incidence, and reproduction number estimates from HealthMap data. The top
  panel shows the weekly incidence derived from HealthMap data and the 6
  weeks incidence forecast on log scale. The solid dots represent the
  observed weekly incidence where the light green
  dots show weeks for which
  all data points were obtained using interpolation. The
  projections are made over 4 week windows, based on the reproduction
  number estimated in the previous 6 weeks. The middle figure in each
  panel shows the reproduction number used to make forecasts over each 4
  week forecast horizon. The bottom figure shows the effective
  reproduction number estimated retrospectively using the full dataset up
  to the length of one calibration window before the end.
  In each case, the solid black line is the median estimate and
  the shaded region represents the 95\% Credible Interval. The red
  horizontal dashed line indicates the $R_t = 1$ threshold. Results are
  shown for the three mainly affected countries although the analysis was
  done jointly using data for all countries in Africa.}
\label{fig:hm64}
\end{figure}\FloatBarrier

\subsubsection{Forecase horizon of 6 weeks}\label{sec:hm66} 

\begin{figure}\centering
\includegraphics[width = \textwidth]{{../../ms-figures/si-figures/HealthMap/projections-viz-fixed-country-15-1.HealthMap_42_42}.pdf}
\caption{Observed and predicted
  incidence, and reproduction number estimates from HealthMap data. The top
  panel shows the weekly incidence derived from HealthMap data and the 6
  weeks incidence forecast on log scale. The solid dots represent the
  observed weekly incidence where the light green
  dots show weeks for which
  all data points were obtained using interpolation. The
  projections are made over 6 week windows, based on the reproduction
  number estimated in the previous 6 weeks. The middle figure in each
  panel shows the reproduction number used to make forecasts over each 6
  week forecast horizon. The bottom figure shows the effective
  reproduction number estimated retrospectively using the full dataset up
  to the length of one calibration window before the end.
  In each case, the solid black line is the median estimate and
  the shaded region represents the 95\% Credible Interval. The red
  horizontal dashed line indicates the $R_t = 1$ threshold. Results are
  shown for the three mainly affected countries although the analysis was
  done jointly using data for all countries in Africa.}
\label{fig:hm66}
\end{figure}\FloatBarrier

\subsubsection{Forecase horizon of 8 weeks}\label{sec:hm68} 

\begin{figure}\centering
\includegraphics[width = \textwidth]{{../../ms-figures/si-figures/HealthMap/projections-viz-fixed-country-15-1.HealthMap_42_56}.pdf}
\caption{Observed and predicted
  incidence, and reproduction number estimates from HealthMap data. The top
  panel shows the weekly incidence derived from HealthMap data and the 6
  weeks incidence forecast on log scale. The solid dots represent the
  observed weekly incidence where the light green
  dots show weeks for which
  all data points were obtained using interpolation. The
  projections are made over 8 week windows, based on the reproduction
  number estimated in the previous 6 weeks. The middle figure in each
  panel shows the reproduction number used to make forecasts over each 8
  week forecast horizon. The bottom figure shows the effective
  reproduction number estimated retrospectively using the full dataset up
  to the length of one calibration window before the end.
  In each case, the solid black line is the median estimate and
  the shaded region represents the 95\% Credible Interval. The red
  horizontal dashed line indicates the $R_t = 1$ threshold. Results are
  shown for the three mainly affected countries although the analysis was
  done jointly using data for all countries in Africa.}
\label{fig:hm68}
\end{figure}\FloatBarrier

\section{Forecasts using WHO data}\label{sec:who}

This section presents the model forecasts over 4 6 and 6 weeks horizon
produced using WHO data and calibration window of 2, 4 or 6 weeks.

\subsection{Calibration window of 2 weeks}\label{sec:who2}
\subsubsection{Forecase horizon of 4 weeks}\label{sec:who24} 

\begin{figure}\centering
\includegraphics[width = \textwidth]{{../../ms-figures/si-figures/WHO/projections-viz-fixed-country-15-1.WHO_14_28}.pdf}
\caption{Observed and predicted
  incidence, and reproduction number estimates from WHO data. The top
  panel shows the weekly incidence derived from WHO data and the 2
  weeks incidence forecast on log scale. The solid dots represent the
  observed weekly incidence. The
  projections are made over 4 week windows, based on the reproduction
  number estimated in the previous 2 weeks. The middle figure in each
  panel shows the reproduction number used to make forecasts over each 4
  week forecast horizon. The bottom figure shows the effective
  reproduction number estimated retrospectively using the full dataset up
  to the length of one calibration window before the end.
  In each case, the solid black line is the median estimate and
  the shaded region represents the 95\% Credible Interval. The red
  horizontal dashed line indicates the $R_t = 1$ threshold. Results are
  shown for the three mainly affected countries although the analysis was
  done jointly using data for all countries in Africa.}
\label{fig:who24}
\end{figure}\FloatBarrier

\subsubsection{Forecase horizon of 6 weeks}\label{sec:who26} 

\begin{figure}\centering
\includegraphics[width = \textwidth]{{../../ms-figures/si-figures/WHO/projections-viz-fixed-country-15-1.WHO_14_42}.pdf}
\caption{Observed and predicted
  incidence, and reproduction number estimates from WHO data. The top
  panel shows the weekly incidence derived from WHO data and the 2
  weeks incidence forecast on log scale. The solid dots represent the
  observed weekly incidence. The
  projections are made over 6 week windows, based on the reproduction
  number estimated in the previous 2 weeks. The middle figure in each
  panel shows the reproduction number used to make forecasts over each 6
  week forecast horizon. The bottom figure shows the effective
  reproduction number estimated retrospectively using the full dataset up
  to the length of one calibration window before the end.
  In each case, the solid black line is the median estimate and
  the shaded region represents the 95\% Credible Interval. The red
  horizontal dashed line indicates the $R_t = 1$ threshold. Results are
  shown for the three mainly affected countries although the analysis was
  done jointly using data for all countries in Africa.}
\label{fig:who26}
\end{figure}\FloatBarrier

\subsubsection{Forecase horizon of 8 weeks}\label{sec:who28} 

\begin{figure}\centering
\includegraphics[width = \textwidth]{{../../ms-figures/si-figures/WHO/projections-viz-fixed-country-15-1.WHO_14_56}.pdf}
\caption{Observed and predicted
  incidence, and reproduction number estimates from WHO data. The top
  panel shows the weekly incidence derived from WHO data and the 2
  weeks incidence forecast on log scale. The solid dots represent the
  observed weekly incidence. The
  projections are made over 8 week windows, based on the reproduction
  number estimated in the previous 2 weeks. The middle figure in each
  panel shows the reproduction number used to make forecasts over each 8
  week forecast horizon. The bottom figure shows the effective
  reproduction number estimated retrospectively using the full dataset up
  to the length of one calibration window before the end.
  In each case, the solid black line is the median estimate and
  the shaded region represents the 95\% Credible Interval. The red
  horizontal dashed line indicates the $R_t = 1$ threshold. Results are
  shown for the three mainly affected countries although the analysis was
  done jointly using data for all countries in Africa.}
\label{fig:who28}
\end{figure}\FloatBarrier


\subsection{Calibration window of 4 weeks}\label{sec:who4}
\subsubsection{Forecase horizon of 4 weeks}\label{sec:who44} 

\begin{figure}\centering
\includegraphics[width = \textwidth]{{../../ms-figures/si-figures/WHO/projections-viz-fixed-country-15-1.WHO_28_28}.pdf}
\caption{Observed and predicted
  incidence, and reproduction number estimates from WHO data. The top
  panel shows the weekly incidence derived from WHO data and the 4
  weeks incidence forecast on log scale. The solid dots represent the
  observed weekly incidence. The
  projections are made over 4 week windows, based on the reproduction
  number estimated in the previous 4 weeks. The middle figure in each
  panel shows the reproduction number used to make forecasts over each 4
  week forecast horizon. The bottom figure shows the effective
  reproduction number estimated retrospectively using the full dataset up
  to the length of one calibration window before the end.
  In each case, the solid black line is the median estimate and
  the shaded region represents the 95\% Credible Interval. The red
  horizontal dashed line indicates the $R_t = 1$ threshold. Results are
  shown for the three mainly affected countries although the analysis was
  done jointly using data for all countries in Africa.}
\label{fig:who44}
\end{figure}\FloatBarrier

\subsubsection{Forecase horizon of 6 weeks}\label{sec:who46} 

\begin{figure}\centering
\includegraphics[width = \textwidth]{{../../ms-figures/si-figures/WHO/projections-viz-fixed-country-15-1.WHO_28_42}.pdf}
\caption{Observed and predicted
  incidence, and reproduction number estimates from WHO data. The top
  panel shows the weekly incidence derived from WHO data and the 4
  weeks incidence forecast on log scale. The solid dots represent the
  observed weekly incidence. The
  projections are made over 6 week windows, based on the reproduction
  number estimated in the previous 4 weeks. The middle figure in each
  panel shows the reproduction number used to make forecasts over each 6
  week forecast horizon. The bottom figure shows the effective
  reproduction number estimated retrospectively using the full dataset up
  to the length of one calibration window before the end.
  In each case, the solid black line is the median estimate and
  the shaded region represents the 95\% Credible Interval. The red
  horizontal dashed line indicates the $R_t = 1$ threshold. Results are
  shown for the three mainly affected countries although the analysis was
  done jointly using data for all countries in Africa.}
\label{fig:who46}
\end{figure}\FloatBarrier

\subsubsection{Forecase horizon of 8 weeks}\label{sec:who48} 

\begin{figure}\centering
\includegraphics[width = \textwidth]{{../../ms-figures/si-figures/WHO/projections-viz-fixed-country-15-1.WHO_28_56}.pdf}
\caption{Observed and predicted
  incidence, and reproduction number estimates from WHO data. The top
  panel shows the weekly incidence derived from WHO data and the 4
  weeks incidence forecast on log scale. The solid dots represent the
  observed weekly incidence. The
  projections are made over 8 week windows, based on the reproduction
  number estimated in the previous 4 weeks. The middle figure in each
  panel shows the reproduction number used to make forecasts over each 8
  week forecast horizon. The bottom figure shows the effective
  reproduction number estimated retrospectively using the full dataset up
  to the length of one calibration window before the end.
  In each case, the solid black line is the median estimate and
  the shaded region represents the 95\% Credible Interval. The red
  horizontal dashed line indicates the $R_t = 1$ threshold. Results are
  shown for the three mainly affected countries although the analysis was
  done jointly using data for all countries in Africa.}
\label{fig:who48}
\end{figure}\FloatBarrier


\subsection{Calibration window of 6 weeks}\label{sec:who6}
\subsubsection{Forecase horizon of 4 weeks}\label{sec:who64} 

\begin{figure}\centering
\includegraphics[width = \textwidth]{{../../ms-figures/si-figures/WHO/projections-viz-fixed-country-15-1.WHO_42_28}.pdf}
\caption{Observed and predicted
  incidence, and reproduction number estimates from WHO data. The top
  panel shows the weekly incidence derived from WHO data and the 6
  weeks incidence forecast on log scale. The solid dots represent the
  observed weekly incidence. The
  projections are made over 4 week windows, based on the reproduction
  number estimated in the previous 6 weeks. The middle figure in each
  panel shows the reproduction number used to make forecasts over each 4
  week forecast horizon. The bottom figure shows the effective
  reproduction number estimated retrospectively using the full dataset up
  to the length of one calibration window before the end.
  In each case, the solid black line is the median estimate and
  the shaded region represents the 95\% Credible Interval. The red
  horizontal dashed line indicates the $R_t = 1$ threshold. Results are
  shown for the three mainly affected countries although the analysis was
  done jointly using data for all countries in Africa.}
\label{fig:who64}
\end{figure}\FloatBarrier

\subsubsection{Forecase horizon of 6 weeks}\label{sec:who66} 

\begin{figure}\centering
\includegraphics[width = \textwidth]{{../../ms-figures/si-figures/WHO/projections-viz-fixed-country-15-1.WHO_42_42}.pdf}
\caption{Observed and predicted
  incidence, and reproduction number estimates from WHO data. The top
  panel shows the weekly incidence derived from WHO data and the 6
  weeks incidence forecast on log scale. The solid dots represent the
  observed weekly incidence. The
  projections are made over 6 week windows, based on the reproduction
  number estimated in the previous 6 weeks. The middle figure in each
  panel shows the reproduction number used to make forecasts over each 6
  week forecast horizon. The bottom figure shows the effective
  reproduction number estimated retrospectively using the full dataset up
  to the length of one calibration window before the end.
  In each case, the solid black line is the median estimate and
  the shaded region represents the 95\% Credible Interval. The red
  horizontal dashed line indicates the $R_t = 1$ threshold. Results are
  shown for the three mainly affected countries although the analysis was
  done jointly using data for all countries in Africa.}
\label{fig:who66}
\end{figure}\FloatBarrier

\subsubsection{Forecase horizon of 8 weeks}\label{sec:who68} 

\begin{figure}\centering
\includegraphics[width = \textwidth]{{../../ms-figures/si-figures/WHO/projections-viz-fixed-country-15-1.WHO_42_56}.pdf}
\caption{Observed and predicted
  incidence, and reproduction number estimates from WHO data. The top
  panel shows the weekly incidence derived from WHO data and the 6
  weeks incidence forecast on log scale. The solid dots represent the
  observed weekly incidence. The
  projections are made over 8 week windows, based on the reproduction
  number estimated in the previous 6 weeks. The middle figure in each
  panel shows the reproduction number used to make forecasts over each 8
  week forecast horizon. The bottom figure shows the effective
  reproduction number estimated retrospectively using the full dataset up
  to the length of one calibration window before the end.
  In each case, the solid black line is the median estimate and
  the shaded region represents the 95\% Credible Interval. The red
  horizontal dashed line indicates the $R_t = 1$ threshold. Results are
  shown for the three mainly affected countries although the analysis was
  done jointly using data for all countries in Africa.}
\label{fig:who68}
\end{figure}\FloatBarrier

\section{Impact of calibration window on model performance}
\label{sec:perftw}

Since changing the length of the calibration window modified the model
complexity with shorter windows introducing more parameters in the
model, we assessed the imapct of this length on the performance of the
model. 

\begin{figure}
  \centering \includegraphics[width=\textwidth]{{/Volumes/Sangeeta EHD/mriids_manuscript/ms-figures/si-figures/ProMED/forecasts-assess-by-time-window-11-1.ProMED_28_assess_by_twindow}.pdf} 
  \caption{Model performance metrics stratified by the time window used for
    model calibration. The performance metrics are
    (A) the percentage of weeks for which the 95\% forecast interval contained
    the observed incidence, (B) relative mean absolute error, (C)
    bias, and (D) sharpness. See methods for details.}
  \label{fig:perftw}
\end{figure}\FloatBarrier


\section{Impact of datasource on model performance}\label{impact-of-datasource-on-model-performance}

\begin{figure}

  \centering 
   \includegraphics[width=\textwidth]{{/Volumes/Sangeeta EHD/mriids_manuscript/ms-figures/si-figures/other/forecasts-assess-by-datasource-7-1.14_28}.pdf} 
  \caption{Model performance metrics stratified by datasource
    ProMED (blue), HealthMap (green), and WHO (yellow).
    The performance metrics are
    (A) the percentage of weeks for which the 95\% forecast interval contained
    the observed incidence, (B) relative mean absolute error, (C)
    bias, and (D) sharpness. See methods for details.}
\label{fig:perfds}
\end{figure}\FloatBarrier


\section{Mobility Model Parameters}\label{mobility-model-parameters}

We estimated the parameters of a gravity model  -
$p_{stay}$ which is the probability of an infectious case staying
within a country, and $\gamma$ which measures the extent to which
distance between two locations influences the flow of people between
them (Fig~\ref{fig:parsul2}).


\begin{figure}
  \includegraphics[width = \textwidth]{/Volumes/Sangeeta EHD/mriids_manuscript/ms-figures/gravity_model_pars_14.pdf}
  \caption{Estimates of mobility model
    parameters during the epidemic. Population movement was modelled using a
    gravity model where the flow between locations \(i\) and \(j\) is
    proportional to the product of their populations and inversely
    population to the distance between them raised to an exponent
    \(\gamma\). The parameter \(\gamma\) thus modulates the influence of
    distance on the population flow. \(p_{stay}\) represents the probability
    of an individual to stay in a given location during their infectious
    period. The solid lines represents the median estimates obtained using
    WHO (yellow), ProMED (blue) and HealthMap (green) data. The shaded
    regions represent the 95\% CrI.}
  \label{fig:parsul2}
\end{figure}\FloatBarrier


\section{Sensitivity Analysis}\label{sec:sensitivity-analysis}

For the results presented in the main text, we choose an uninformative
uniform prior for the parameter \(\gamma\) with an upper bound 2. We
also fitted the model with a uniform prior for \(\gamma\) allowing it to
vary from 1 to 10. In this section we present the results from the
sensitivity analyses. Fig~\ref{fig:parsul2} presents the
estimates of the parameters over the course of the epidemic using the
alternative priors. As the epidemic progressed, 
the parameter $p_{stay}$ assumed larger values suggesting a decreased
probability of travel over time (Figures~\ref{fig:parsul2},
\ref{fig:parsul10}). As $p_{stay}$ assumes large values,
the estimated flow is more strongly influenced by $p_{stay}$ than by
$\gamma$. Furthermore, $p_{stay}$ is likely to depend on the spatial
scale of the model. Our analyses were carried out at the
national scale; we expect that $\gamma$ will be more sensitive to
$p_{stay}$ at a finer spatial resolution. Overall, the  flow
between locations using the parameters estimated using the two
alternative priors did not vary much
\ref{fig:flows2}. Figures~\ref{fig:pm24ul10} to \ref{fig:pm68ul10}
present the model forecasts using the alternative priors for $\gamma$
and Figure~\ref{fig:perfbygamma} presents a comparison of model
performance metrics using the two priors.
Although the analysis was carried out for the
three data sources (ProMED, HealthMap and WHO), for brevity we present
results using ProMED data only.


\begin{figure}
  \centering 
   \includegraphics[width = \textwidth]{/Volumes/Sangeeta EHD/mriids_manuscript/ms-figures/si-figures/ProMED/sensitivity_analyses/2019-10-28_gravity_model_pars_14}
  \caption{Estimates of mobility model
  parameters during the epidemic. Population movement was modelled using a
  gravity model where the flow between locations \(i\) and \(j\) is
  proportional to the product of their populations and inversely
  population to the distance between them raised to an exponent \(gamma\).
  The parameter gamma thus modulates the influence of distance on the
  population flow. Here \(\gamma\) is allowed to vary between 1 and 10.
  \(p_{stay}\) represents the probability of an individual to stay in a
  given location during their infectious period. The solid lines
  represents the median estimates obtained using ProMED data. The shaded
  regions represent the 95\% CrI.}
\label{fig:parsul10}
\end{figure}\FloatBarrier

\begin{figure}
\centering
   \includegraphics[width = \textwidth]{/Volumes/Sangeeta
     EHD/mriids_manuscript/ms-figures/si-figures/ProMED/sensitivity_analyses/2019-10-28_flow_using_diff_gravity_model_pars}
\caption{Estimated flows using gravity model. The x-axis shows the flows using a uniform
  prior for $\gamma$ with upper limit 2 and the y-axis shows the flows
  using a uniform prior varying from 1 to 10. The black dots show the
  flows estimated using the first 21 days of incidence data from ProMED. Flows
  estimated using parameters fitted to the first 210 days of incidence
  data are shown in red. Results are shown for the model with
  calibration window set to 14 days.}
\label{fig:flows2}
\end{figure}\FloatBarrier

\subsection{Forecasts using ProMED data}\label{forecasts-using-promed-data-1}
\subsubsection{Calibration window of 2 weeks}\label{sec:pm2-3-2}
\subsubsection{Forecast horizon 4 weeks}\label{forecast-horizon-4-weeks-8}

\begin{figure}
  \centering 
\includegraphics[width = \textwidth]{/Volumes/Sangeeta EHD/mriids_manuscript/ms-figures/si-figures/ProMED/sensitivity_analyses/projections-viz-fixed-country-15-1.ProMED_14_28} 
  \caption{Observed and predicted incidence, and reproduction number
estimates from ProMED data. The calibration window is 2 weeks and the
forecast horizon is 4 weeks.}
\label{fig:pm24ul10}
\end{figure}\FloatBarrier


\subsubsection{Forecast horizon 6 weeks}\label{forecast-horizon-6-weeks-9}

\begin{figure}

  \centering \includegraphics[width = \textwidth]{/Volumes/Sangeeta EHD/mriids_manuscript/ms-figures/si-figures/ProMED/sensitivity_analyses/projections-viz-fixed-country-15-1.ProMED_14_42} 

  \caption{Observed and predicted incidence, and reproduction number
    estimates from ProMED data.The calibration window is 2 weeks and
    the forecast horizon is 6 weeks.}
  \label{fig:pm26ul10}
\end{figure}\FloatBarrier


\subsubsection{Forecast horizon 8 weeks}\label{forecast-horizon-8-weeks-9}

\begin{figure}

  \centering 
\includegraphics[width = \textwidth]{/Volumes/Sangeeta EHD/mriids_manuscript/ms-figures/si-figures/ProMED/sensitivity_analyses/projections-viz-fixed-country-15-1.ProMED_14_56} 
  \caption{Observed and predicted incidence, and reproduction number
    estimates from ProMED data.The calibration window is 2 weeks and
    the forecast horizon is 8 weeks.}
\label{fig:pm28ul10}
\end{figure}\FloatBarrier


\subsubsection{Calibration window of 4 weeks}\label{calibration-window-of-4-weeks-3}
\subsubsection{Forecast horizon 4 weeks}\label{forecast-horizon-4-weeks-9}

\begin{figure}
    \centering \includegraphics[width = \textwidth]{/Volumes/Sangeeta EHD/mriids_manuscript/ms-figures/si-figures/ProMED/sensitivity_analyses/projections-viz-fixed-country-15-1.ProMED_28_28} 
  \caption{Observed and predicted incidence, and reproduction number
    estimates from ProMED data.The calibration window is 4 weeks and
    the forecast horizon is 4 weeks.}
  \label{fig:pm44ul10}
\end{figure}\FloatBarrier

\subsubsection{Forecast horizon 6 weeks}\label{forecast-horizon-6-weeks-10}

\begin{figure}
  \centering \includegraphics[width = \textwidth]{/Volumes/Sangeeta EHD/mriids_manuscript/ms-figures/si-figures/ProMED/sensitivity_analyses/projections-viz-fixed-country-15-1.ProMED_28_42} 
  \caption{Observed and predicted incidence, and reproduction number
    estimates from ProMED data.The calibration window is 4 weeks and
    the forecast horizon is 6 weeks.}
  \label{fig:pm46ul10}
\end{figure}\FloatBarrier


\subsubsection{Forecast horizon 8 weeks}\label{forecast-horizon-8-weeks-10}
\begin{figure}
  \centering
    \includegraphics[width = \textwidth]{/Volumes/Sangeeta EHD/mriids_manuscript/ms-figures/si-figures/ProMED/sensitivity_analyses/projections-viz-fixed-country-15-1.ProMED_28_56} 
  \caption{Observed and predicted incidence, and reproduction number
    estimates from ProMED data.The calibration window is 4 weeks and
    the forecast horizon is 8 weeks.}
  \label{fig:pm48ul10}
\end{figure}\FloatBarrier


\subsubsection{Calibration window of 6 weeks}\label{calibration-window-of-6-weeks-3}


\subsubsection{Forecast horizon 4 weeks}\label{forecast-horizon-4-weeks-10}

\begin{figure}  
    \centering 
\includegraphics[width = \textwidth]{/Volumes/Sangeeta EHD/mriids_manuscript/ms-figures/si-figures/ProMED/sensitivity_analyses/projections-viz-fixed-country-15-1.ProMED_42_28} 
  \caption{Observed and predicted incidence, and reproduction number
    estimates from ProMED data.The calibration window is 6 weeks and
    the forecast horizon is 4 weeks.}
  \label{fig:pm64ul10 }
\end{figure}\FloatBarrier


\subsubsection{Forecast horizon 6 weeks}\label{forecast-horizon-6-weeks-11}

\begin{figure}

  \centering 
\includegraphics[width = \textwidth]{/Volumes/Sangeeta EHD/mriids_manuscript/ms-figures/si-figures/ProMED/sensitivity_analyses/projections-viz-fixed-country-15-1.ProMED_42_42} 
  \caption{Observed and predicted incidence, and reproduction number
    estimates from ProMED data.The calibration window is 6 weeks and
    the forecast horizon is 6 weeks.}
  \label{fig:pm66ul10}
\end{figure}\FloatBarrier


\subsubsection{Forecast horizon 8 weeks}\label{forecast-horizon-8-weeks-11}

\begin{figure}
    \centering
    \includegraphics[width = \textwidth]{/Volumes/Sangeeta EHD/mriids_manuscript/ms-figures/si-figures/ProMED/sensitivity_analyses/projections-viz-fixed-country-15-1.ProMED_42_56} 
  \caption{Observed and predicted incidence, and reproduction number
    estimates from ProMED data.The calibration window is 6 weeks and
    the forecast horizon is 8 weeks.}
  \label{fig:pm68ul10}
\end{figure}\FloatBarrier


\subsection{Model performance with alternate priors for mobility
  model parameter}\label{model-performance-with-alternate-priors-for-mobility-model-parameter}

\begin{figure}
    \centering
    \includegraphics[width = \textwidth]{/Volumes/Sangeeta EHD/mriids_manuscript/ms-figures/si-figures/ProMED/sensitivity_analyses/2019-10-24_ProMED_forecasts_assess_by_gamma_ul_14_28.pdf} 
  \caption{Model performance metrics allowing $\gamma$ to vary from
    1 to 2 or 10. The performance metrics (anti-clockwise from top left) are
    bias, sharpness, relative mean absolute error (show on log scale) and
    the percentage of weeks for which the 95% forecast interval contained
    the observed incidence.}\label{fig:perfbygamma}
\end{figure}\FloatBarrier

\section{Risk of spatial spread}\label{sec:spatial-spread}

In this section, we present additional analyses carried out on
predicting the spatial spread of the epidemic. Classification of
alerts raised 1, 2, 3, and 4 weeks ahead are shown in
Fig~\ref{fig:alerts4weekahead}. We also assessed the sensitivity
of the model when the analysis was restricted to countries other than
Guinea, Liberia and Sierra Leone (Fig~\ref{fig:alertsallbut3}). At
50\% threshold, the
model exhibited high specificity (98.0\%) but poor sensitivity (42.9\%) in predicting 
presence of cases in weeks following a week with no observed cases in each
country \ref{fig:rocnocases}. In predicting presence of cases in
countries with no or low incidence, or in a
week following a week in which no cases were observed, the sensitivity
improved at higher thresholds with a reasonably low false alert rate
(Fig~\ref{fig:tprbythreshold}). Finally, we find that the model is
able to attain a high sensitivity and specificity relatively early in 
the epidemic (Fig~\ref{fig:tprovertime}).
 
\begin{figure}
\centering
\includegraphics[width = \textwidth]{/Volumes/Sangeeta EHD/mriids_manuscript/ms-figures/si-figures/ProMED/spatial_spread/2019-10-23_alerts_by_week_ProMED_14_28}
\caption{4 weeks ahead alerts}
\label{fig:alerts4weekahead}
\end{figure}\FloatBarrier


\begin{figure}
\centering
\includegraphics[width = \textwidth]{/Volumes/Sangeeta EHD/mriids_manuscript/ms-figures/si-figures/ProMED/spatial_spread/2019-10-28_alerts_in_1_all_roc_ProMED_14_28_allbut3}
\caption{4 weeks ahead alerts}
\label{fig:alertsallbut3}
\end{figure}\FloatBarrier

\begin{figure}
\centering
\begin{subfigure}[b]{0.45\textwidth}
\includegraphics[width = \textwidth]{/Volumes/Sangeeta EHD/mriids_manuscript/ms-figures/si-figures/ProMED/spatial_spread/2019-10-28_roc_ProMEDno_cases_last_week_14_28}
\end{subfigure}
\begin{subfigure}[b]{0.45\textwidth}
\includegraphics[width = \textwidth]{/Volumes/Sangeeta
  EHD/mriids_manuscript/ms-figures/si-figures/ProMED/spatial_spread/2019-10-29_alerts_in_1_ProMED_no_cases_last_week_14_28}
\end{subfigure}
\caption{ROC no cases last week}
\label{fig:rocnocases}
\end{figure}\FloatBarrier



\begin{figure}
\centering
\begin{subfigure}[b]{0.45\textwidth}
\includegraphics[width = \textwidth]{/Volumes/Sangeeta EHD/mriids_manuscript/ms-figures/si-figures/ProMED/spatial_spread/2019-10-23_ProMED_tpr_by_threshold_14_28_all_but_3}
\end{subfigure}
\begin{subfigure}[b]{0.45\textwidth}
\includegraphics[width = \textwidth]{/Volumes/Sangeeta
  EHD/mriids_manuscript/ms-figures/si-figures/ProMED/spatial_spread/2019-10-29_ProMED_tpr_by_threshold_14_28_no_cases_last_week}
\end{subfigure}
\caption{True and False alert rates at various thresholds for all
  countries except Guinea, Liberia and Sierra Leone.}
\label{fig:tprbythreshold}
\end{figure}\FloatBarrier


\begin{figure}
\centering
\includegraphics[width = \textwidth]{/Volumes/Sangeeta EHD/mriids_manuscript/ms-figures/si-figures/other/2019-10-23_tpr_fpr_by_date_obs_ds_14_28}
\caption{True and False alert rates over the course of the epidemic.}
\label{fig:tprovertime}
\end{figure}\FloatBarrier

\end{document}